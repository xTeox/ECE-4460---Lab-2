\subsection{Timer B ISR without and with RTOS}

The main purpose of this procedure is to develop a program that counts Timer B interrupts and displays the interrupt overhead and after a RTOS program with the same functionalities added to the simulation of a PWM signal with a LED.

Appendix C shows the program code without RTOS with the proper special timer function register set up. Appendix D shows the program code with RTOS. This program uses the same structure of Part B with one event for a timer semaphore and two tasks and four simple procedures as follow:

\begin{itemize}[label={--}]
\item TaskStart(): Task function that startup other tasks and update screen with its task's statistic;
\item Real-time operational system kernel μC/OS-II;
\item Codes provided by the laboratory instructions and the official course website.
\item TaskScreenUpd(): Task function that updates screen with Timer B interrupts and overhead;
\item InitTimerInt(): Procedure setup Timer B interrupts;
\item Tmr_B_isr(): Procedure holds the IISR;
\item UpdateStat(): Procedure update RTOS statistics;
\item DispStr(): Procedure set the STDIO cursor location and display a string in the screen.
\end{itemize}

All Timer B special function register were set properly to enable its clock with perclk/16, priority 1, and B1 count register of 0x15E, as required. The IISR reads the timer counter, increments the necessary variables, post the semaphore when the count reaches 1000 in order to inform the function TaskScreenUpd() that updates screen with the number of interrupts happened and the timer overhead. This last value is calculated by the difference of the timer counter and the count register predefined.

As the Rabbit 2000 is a 8-bit microprocessor \cite{Digi_International_2007}, the 10-bit Timer B count is divided into two 8-bit timer counts: TBCMR 2 MSBs and TBCLR 8 bits, one after another. Because of this, if one of these counters is read after the other, it can be changed between this process. Thus, to calculate correctly this combined value, TBCLR is read, then TBCMR, and after TBCLR again to check if there is any change in its MSB. If there is, TBCMR is decreased. Finally, the timer counter value Timer_B_value is set with the value of TBCLR summed to TBCMR shifted 2 bits to the left. 

In order to be able to simulate the PWM signal, it is needed the correct value of the system frequency. Thus, to avoid problems with possible trades of development boards, this value were captured again following the procedure of the last laboratory's first part. However, it was noticed that the method used less information than it was necessary. Therefore, the procedure was redone capturing the correct information about the amount of cycles is necessary to complete the code and then calculate correctly the CPU frequency. The value returned in this procedure was 128 cycles and approximately 20.9 GHz. This frequency was defined as SYSTEM_CLOCK in the code and then used in the other constants to help in the calculation of the duty timer count to simulate the PWM signal.