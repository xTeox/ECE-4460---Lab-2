\subsection{Timer B ISR without and with RTOS}

The main purpose of this procedure is to develop a program that counts Timer B interrupts and displays the interrupt overhead and after a RTOS program with the same functionalities added to the simulation of a PWM signal with a LED.

In order to be able to simulate the PWM signal, it is needed the correct value of the system frequency. Thus, to avoid problems with possible trades of development boards, this value were captured again following the procedure of the last laboratory's first part. However, it was noticed that the method used less information than it was necessary. Therefore, the procedure was redone capturing the correct information about the amount of cycles is necessary to complete the code and then calculate correctly the CPU frequency. The value returned in this procedure was 128 cycles and approximately 20.9 GHz. This frequency was defined as SYSTEM_CLOCK in the code and then used in the other constants to help in the calculation of the duty timer count to simulate the PWM signal.