\section{Modified code LAB2B.c.}
\begin{lstlisting}[language=C]
#define TASK_STK_SIZE 512
#define N_TASKS         3
#define MAX_COUNT     100
#define TMA5_MASK    0x20
#define TMA6_MASK    0x40
#define TMA7_MASK    0x80
#define OS_MAX_EVENTS      2
#define OS_MAX_TASKS      11
#define OS_TASK_STAT_EN    1
#define OS_TICKS_PER_SEC 128
#use "ucos2.lib"
void InitTimerInt();
void Tmr_A_isr();
void UpdateStat();
void Task(void *data);
void TaskStart(void *data);
void DispStr(int x, int y, char *s);
char count_TMRA5;
char count_TMRA6;
char count_TMRA7;
int display_count[N_TASKS];
UBYTE     TaskData[N_TASKS];
OS_EVENT *TimerSem[N_TASKS];
void main (void)
{
    auto UBYTE i;
    brdInit();
    OSInit();
    OSTaskCreate(TaskStart, (void*) 0,
                 TASK_STK_SIZE, 10);
    for (i = 0 ; i < N_TASKS ; i++)
        TimerSem[i] = OSSemCreate(0);
    OSStart();
}
void TaskStart (void *data)
{
    auto UBYTE  i;
    for (i = 0; i < N_TASKS; i++) {
        TaskData[i] = i;
        OSTaskCreate(Task, (void *)&TaskData[i],
                     TASK_STK_SIZE, 11+i);
    }
    InitTimerInt();
    OSStatInit();
    DispStr(0, 12,
     "#Tasks          : xxxxx  CPU Usage: xxxxx %");
    DispStr(0, 13, "#Task switch/sec: xxxxx");
    DispStr(0, 14, "<-PRESS 'Q' TO QUIT->");
    for (;;) {
        UpdateStat();
        OSTimeDly(OS_TICKS_PER_SEC);
    }
}
nodebug void Task (void *data)
{
    auto UBYTE err;
    auto UBYTE num;
    auto char display[128];
    num = *((int *)data) + 5;
    sprintf(display,
     "Number of TMRA%d interrupts: xxxxx  (x100)",
     num);
    DispStr(20, num, display);
    for (;;) {
        num = *((int *)data);
        display[0] = '\0';
        OSSemPend(TimerSem[num], 0, &err);
        sprintf(display, "%5d", display_count[num]);
        DispStr(48, num+5, display);
    }
}
nodebug root interrupt void Tmr_A_isr()
{
    char TMRA_status;
    TMRA_status = RdPortI(TACSR);
    if (TMRA_status & TMA7_MASK) {
        count_TMRA7++;
        if (count_TMRA7 == MAX_COUNT) {
            display_count[2]++;
            count_TMRA7 = 0;
            OSSemPost(TimerSem[2]);
        }
    }
    if (TMRA_status & TMA6_MASK) {
        count_TMRA6++;
        if (count_TMRA6 == MAX_COUNT) {
            display_count[1]++;
            count_TMRA6 = 0;
            OSSemPost(TimerSem[1]);
        }
    }
    if (TMRA_status & TMA5_MASK) {
        count_TMRA5++;
        if (count_TMRA5 == MAX_COUNT) {
            display_count[0]++;
            count_TMRA5 = 0;
            OSSemPost(TimerSem[0]);
        }
    }
    OSIntExit();
}
void InitTimerInt()
{
    auto UBYTE i;
    count_TMRA5 = 0;
    count_TMRA6 = 0;
    count_TMRA7 = 0;
    for (i = 0; i < N_TASKS; i++)
        display_count[i] = 0;
    SetVectIntern(0x0A, Tmr_A_isr);
    WrPortI(TAT7R, &TAT7RShadow, 0x3F);
    WrPortI(TAT6R, &TAT6RShadow, 0x7F);
    WrPortI(TAT5R, &TAT5RShadow, 0xFF);
    WrPortI(TAT1R, &TAT1RShadow, 0xFF);
    WrPortI(TACR,  &TACRShadow,  0xE1);
    WrPortI(TACSR, &TACSRShadow, 0xE1);
}
void UpdateStat()
{
    char key, s[50];
    sprintf(s, "%5d", OSTaskCtr);
    DispStr(18, 12, s);
    #if OS_TASK_STAT_EN
        sprintf(s, "%5d", OSCPUUsage);
        DispStr(36, 12, s);
    #endif
    sprintf(s, "%5d", OSCtxSwCtr);
    DispStr(18, 13, s);
    OSCtxSwCtr = 0;
    if (kbhit()) {
        key = getchar();
        if (key == 0x71 || key == 0x51)
            exit(0);
    }
}
void DispStr (int x, int y, char *s)
{
    x += 0x20;
    y += 0x20;
    printf ("\x1B=%c%c%s", x, y, s);
}
\end{lstlisting}
\\ 
\\ 
\\ 
\\ 
\\ 