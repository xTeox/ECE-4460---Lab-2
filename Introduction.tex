\section{Introduction}

The purpose of this lab is to become familiar with internal interrupt service routines (IISRs), aka synchronous interrupts, by using the kernel of the real-time operational system (RTOS) μC/OS-II in the IDE environment of Dynamic C to program in a Rabbit 2000 microprocessor unit of a development board Wildcat BL2000. Also, a sample of code to the configuration of a simple real-time operational system (RTOS) using μC/OS-II is provided.

\subsection{Objectives}
The experiment is divided into three procedures, each with its own objectives of developing:

\end{enumerate}
\begin{enumerate}[A.]
\item
\textit{Timer A ISR}
\\  A simple program that displays the number of times each of the Timer A interrupts has occurred on the computer monitor.
\\
\item
\textit{Timer A ISR with RTOS}
\\ A RTOS that holds the same functionality of the previous program but without using 100\% of the CPU.
\\
\item
\textit{Timer B ISR with RTOS}
\\ A RTOS that counts Timer B interrupts and displays the interrupt overhead.
\\
\end{enumerate}