\section{Introduction}

The purpose of this lab is to become familiar with internal interrupt service routines (IISRs), aka synchronous interrupts, by using the kernel of the real-time operational system (RTOS) μC/OS-II in the IDE environment of Dynamic C to program in a Rabbit 2000 microprocessor unit of a development board Wildcat BL2000. Also, a sample of code to the configuration of a simple real-time operational system using μC/OS-II is provided.

\subsection{Objectives}
The experiment is divided into three procedures, each with its own objectives as follows:

\begin{enumerate}[(i)]
\item
\textit{Timer A ISR}\\
Displays the number of times each of the Timer A interrupts have occured on the computer monitor.
\\
\item
\textit{Timer B ISR}\\
Displays the number of times each of the Timer B interrupts have occured on the computer monitor.
\\
\item
\textit{Timer B ISR with reduce the CPU load}
\\Use the same structure from the previous procedure to read values from a sensor and write signal to a PWM to control and visualize the position of a ball inside a tube.
\\
\end{enumerate}