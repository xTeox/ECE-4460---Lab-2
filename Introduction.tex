\section{Introduction}

The purpose of this lab is to become familiar with internal interrupt service routines (IISRs), aka synchronous interrupts, by using the kernel of the real-time operational system (RTOS) μC/OS-II in the IDE environment of Dynamic C to program in a Rabbit 2000 microprocessor unit of a development board Wildcat BL2000. Also, a sample of code to the configuration of a simple real-time operational system using μC/OS-II is provided.

\subsection{Objectives}
The experiment is divided into three procedures, each with its own objectives as follows:

\begin{enumerate}[(i)]

\item
\textit{Main Clock Frequency}
\\Measure the execution time of a code and the clock frequency of the system using a simple program with a loop;

\item
\textit{Multitask Program using \mu C/OS-II}
\\Introduce the basic steps of creating and measuring CPU details of a real-time operational system using \mu C/OS-II and illustrate the multitasking with two LED blinkers with different frequencies;

\item
\textit{Analog to Digital Conversion}
\\Use the same structure from the previous procedure to read values from a sensor and write signal to a PWM to control and visualize the position of a ball inside a tube.

\end{itemize}

\begin{lstlisting}[frame=single]
\end{lstlisting}