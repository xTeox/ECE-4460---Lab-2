\begin{appendices}
\section{Appendix A: \\Modified code LAB2A.c.}
\begin{lstlisting}[language=C]
#define MAX_COUNT   100
#define TMA5_MASK  0x20
#define TMA6_MASK  0x40
#define TMA7_MASK  0x80
char count_TMRA5;
char count_TMRA6;
char count_TMRA7;
int  display_count[3];
char i;
void Tmr_A_isr();
void DispStr(int x, int y, char *s);
main()
{
    auto char display[128];
    auto char key;
    brdInit();
    count_TMRA5 = 0;
    count_TMRA6 = 0;
    count_TMRA7 = 0;
    for (i = 0; i < 3; i++)
        display_count[i] = 0;
    SetVectIntern(0x0A, Tmr_A_isr);
    WrPortI(TAT7R, &TAT7RShadow, 0x3F);
    WrPortI(TAT6R, &TAT6RShadow, 0x7F);
    WrPortI(TAT5R, &TAT5RShadow, 0xFF);
    WrPortI(TAT1R, &TAT1RShadow, 0xFF);
    WrPortI(TACR,  &TACRShadow,  0xE1);
    WrPortI(TACSR, &TACSRShadow, 0xE1);
    DispStr(5, 10, "<-PRESS 'Q' TO QUIT->");
    while (1) {
        for(i = 0; i < 3; i++) {
            display[0] = '\0';
            sprintf(display, 
             "Number of TMRA%d interrupts: %d x100",
             i+5, display_count[i]);
            DispStr(5, i+1, display);
        }
        if (kbhit()) {
            key = getchar();
            if (key == 0x71 || key == 0x51)
                exit(0);
        }
    }
}
nodebug root interrupt void Tmr_A_isr()
{
    char TMRA_status;
    TMRA_status = RdPortI(TACSR);
    if (TMRA_status & TMA5_MASK) {
        count_TMRA5++;
        if (count_TMRA5 == MAX_COUNT) {
            display_count[0]++;
            count_TMRA5 = 0;
        }
    }
    if (TMRA_status & TMA6_MASK) {
        count_TMRA6++;
        if (count_TMRA6 == MAX_COUNT) {
            display_count[1]++;
            count_TMRA6 = 0;
        }
    }
    if (TMRA_status & TMA7_MASK) {
        count_TMRA7++;
        if (count_TMRA7 == MAX_COUNT) {
            display_count[2]++;
            count_TMRA7 = 0;
        }
    }
}
void DispStr(int x, int y, char *s)
{
   x += 0x20;
   y += 0x20;
   printf ("\x1B=%c%c%s", x, y, s);
}
\end{lstlisting}