\begin{lstlisting}[language=C++]
/* LAB 2 Part A */
/* This program displays the number of times each of the Timer A interrupts have occured on the computer monitor */

/* Definitions */
#define	MAX_COUNT	100									// Maximum Count Value
#define TMA5_MASK    0x20									// Mask for Timer A5 in TACSR
#define TMA6_MASK		0x40									// Mask for Timer A6 in TACSR
#define TMA7_MASK		0x80									// MAsk for Timer A7 in TACSR

/* Variable declarations */
char count_TMRA5;														// This variable will count the number of times Timer A5 has triggered an interrupted
char count_TMRA6;														// This variable will count the number of times Timer A6 has triggered an interrupted
char count_TMRA7;														// This variable will count the number of times Timer A7 has triggered an interrupted
int display_count[3];												// This variable will hold the count values divided by 100
char i;																	// iterator

/* function prototype */
void Tmr_A_isr(); 													// Timer A interrupt service routine
void DispStr(int x, int y, char *s);							// display strings on computer monitor

main()
{
	/* Initializations */
	auto char display[128];
   auto char key;

	brdInit();
	count_TMRA5 = 0;													// initialize counters to zero
	count_TMRA6 = 0;													// initialize counters to zero
	count_TMRA7 = 0;													// initialize counters to zero

   for (i = 0; i < 3; i++)											// initialize display counters to zero
		display_count[i] = 0;


	/*************** FILL IN CODE TO REPLACE '------'. CODE SHOULD BE CONSISTENT WITH COMMENTS. REFER TO FIGURE 5. IN LAB MANUAL. *******************/
	/* set up timer A interrupt */
	SetVectIntern(0x0A, Tmr_A_isr); 									// set up timer A interrupt vector
	WrPortI(TAT7R, &TAT7RShadow, 0x3F);								// set TMRA7 to count down from 63
  	-----------------------------------								// set TMRA6 to count down from 127
	-----------------------------------								// set TMRA5 to count down from 255
	-----------------------------------								// set TMRA1 to count down from 255
	WrPortI(TACR, &TACRShadow, ----);								// Clock TMRA5,TMRA6,TMRA7 by TMRA1. Clock TMRA4 by PCLK/2. Set Timer A to trigger priority 1 interrupts
	WrPortI(TACSR, &TACSRShadow,----); 								// Enable interrupts for TMRA5,TMRA6,TMRA7 only. Enable main clock for Timer A
	/************************************************************************************************************************************************/


   DispStr(5, 10, "<-PRESS 'Q' TO QUIT->");

	/* This is the background process. It repeatedly updates the computer monitor with the values in the display_count array */
	while(1)	{
		for(i = 0; i < 3; i++)	{
			display[0] = '\0';
			sprintf(display, "Number of TMRA%d interrupts: %d\  (x100)",i+5, display_count[i]);
			DispStr(5, i+1, display);
		}
      if (kbhit()) {											// See if key has been pressed
			key = getchar();
			if (key == 0x71 || key == 0x51)				// Yes, see if it's the q or Q key
				exit(0);
         }
 		}
}


/* Timer A interrupt service routine */
nodebug root interrupt void Tmr_A_isr()
{
	char TMRA_status;														// This will hold the interrupt flags
	TMRA_status = RdPortI(TACSR);										// Read interrupt flags and store to TMRA_status

/* There are three possible events leading to this service routine being called. Write code to determine which of the three
timers has triggered the interrupt. Note it is possible for more than one timer to expire at the same time, thus your code must service
all triggers. By testing the interrupt flags in TMRA_status, you can determine which counters have expired. Use
three count variables count_TMRAX (X is 5, 6 and 7) to count the number of interrupts. For each Timer AX interrupt you must increment the corresponding count variable.
i.e.,	if Timer A5 has expired then increment count_TMRA5. */

/*** During your lab session, you will replace the following code to perform as specified above ***/
count_TMRA5++;
count_TMRA6++;
count_TMRA7++;
/**************************************************************************************************/


	/* The following code increment the display array for every 100 interrupts for a respective timer, and resets the count variable */

	if (count_TMRA5 == MAX_COUNT){									// if Timer A5 has interrupted the CPU 100 times
		display_count[0]++;												// increment the display counter
		count_TMRA5 = 0;													// clear the Timer A5 counter
	}

	if (count_TMRA6 == MAX_COUNT){									// if Timer A6 has interrupted the CPU 100 times
		display_count[1]++;												// increment the display counter
		count_TMRA6 = 0;													// clear the Timer A6 counter
	}

	if (count_TMRA7 == MAX_COUNT){									// if Timer A6 has interrupted the CPU 100 times
		display_count[2]++;												// increment the display counter
		count_TMRA7 = 0;													// clear the Timer A6 counter
	}
}

/* set the STDIO cursor location and display a string  */
void DispStr(int x, int y, char *s)
{
   x += 0x20;
   y += 0x20;
   printf ("\x1B=%c%c%s", x, y, s);
}
\end{lstlisting}

\textit{Oh, an empty article!} 

You can get started by \textbf{double clicking} this text block and begin editing. You can also click the \textbf{Text} button below to add new block elements. Or you can \textbf{drag and drop an image} right onto this text. Happy writing!