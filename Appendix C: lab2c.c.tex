\vfill \columnbreak
\section{Modified code LAB2C.c.}
\begin{lstlisting}[language=C]
int int_count;
int display_int_count;
int Timer_B_value;
void Tmr_B_isr();
void DispStr(int x, int y, char *s);
main()
{
    auto char display[128];
    auto char key;
    brdInit();
    int_count = 0;
    display_int_count = 0;
    Timer_B_value = 0;
    SetVectIntern(0x0B, Tmr_B_isr);
    WrPortI(TBCR, &TBCRShadow, 0x0A);
    WrPortI(TBM1R, NULL, 0x40 );
    WrPortI(TBL1R, NULL, 0x5E    );
    WrPortI(TBCSR, &TBCSRShadow, 0x03 );
    while(1) {
        display[0] = '\0';
        sprintf(display, 
         "Number of TimerB interrupts = %d (x1000)",
         display_int_count);
        DispStr(5, 2, display);
        display[0] = '\0';
        sprintf(display, 
         "TimerB overhead = %d  (xPCLK/16 cycles)",
         Timer_B_value - 350);
        DispStr(5, 4, display);
        if (kbhit()) {
            key = getchar();
            if (key == 0x71 || key == 0x51)
                exit(0);
        }
    }
}
nodebug root interrupt void Tmr_B_isr()
{
    char TMRB_status;
    int tl, th;
    TMRB_status = RdPortI(TBCSR);
    WrPortI(TBM1R, NULL, 0x40);
    WrPortI(TBL1R, NULL, 0x5E);
    int_count++;
    if (int_count == 1000){
        display_int_count++;
        int_count = 0;
    }
    tl = RdPortI(TBCLR);
    th = RdPortI(TBCMR) & 0xC0;
    if (!(tl & 0x80) & (tl ^ RdPortI(TBCLR)) & 0x80)
        th = RdPortI(TBCMR)-1;
    Timer_B_value = (((int)th & 0xC0) << 2) + tl;
}
void DispStr(int x, int y, char *s)
{
    x += 0x20;
    y += 0x20;
    printf ("\x1B=%c%c%s", x, y, s);
}
\end{lstlisting}